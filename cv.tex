%% start of file `template.tex'.
%% Copyright 2006-2013 Xavier Danaux (xdanaux@gmail.com).
%
% This work may be distributed and/or modified under the
% conditions of the LaTeX Project Public License version 1.3c,
% available at http://www.latex-project.org/lppl/.


\documentclass[11pt,a4paper,roman]{moderncv}        % possible options include font size ('10pt', '11pt' and '12pt'), paper size ('a4paper', 'letterpaper', 'a5paper', 'legalpaper', 'executivepaper' and 'landscape') and font family ('sans' and 'roman')

% modern themes
\moderncvstyle{banking}                            % style options are 'casual' (default), 'classic', 'oldstyle' and 'banking'
\moderncvcolor{blue}                                % color options 'blue' (default), 'orange', 'green', 'red', 'purple', 'grey' and 'black'
%\renewcommand{\familydefault}{\sfdefault}         % to set the default font; use '\sfdefault' for the default sans serif font, '\rmdefault' for the default roman one, or any tex font name
\nopagenumbers{}                                  % uncomment to suppress automatic page numbering for CVs longer than one page

% character encoding
\usepackage[utf8]{inputenc}
\usepackage{fontawesome}
\usepackage{tabularx}
\usepackage{ragged2e}
%\usepackage{CJKutf8}                              % if you need to use CJK to typeset your resume in Chinese, Japanese or Korean

% adjust the page margins
\usepackage[scale=0.8]{geometry}
\usepackage{multicol}
%\setlength{\hintscolumnwidth}{3cm}                % if you want to change the width of the column with the dates
%\setlength{\makecvtitlenamewidth}{10cm}           % for the 'classic' style, if you want to force the width allocated to your name and avoid line breaks. be careful though, the length is normally calculated to avoid any overlap with your personal info; use this at your own typographical risks...

\usepackage{import}

% personal data
\name{Jack}{Kennedy}
% \title{Curriculum Vitae}                               % optional, remove / comment the line if not wanted
\address{Newcastle Upon Tyne, United Kingdom }{PhD Student in Bayesian Statistics with application to renewable energy}{}% optional, remove / comment the line if not wanted; the "postcode city" and and "country" arguments can be omitted or provided empty
% \phone[mobile]{909-839-3097}                   % optional, remove / comment the line if not wanted
% \phone[fixed]{01234 123456}                    % optional, remove / comment the line if not wanted
%\phone[fax]{+3~(456)~789~012}                      % optional, remove / comment the line if not wanted
% \email{xpan1@swarthmore.edu}                               % optional, remove / comment the line if not wanted
% \homepage{shawnpan.me}                         % optional, remove / comment the line if not wanted
 %\extrainfo{}          % optional, remove / comment the line if not wanted
%\photo[64pt][0.4pt]{picture}                       % optional, remove / comment the line if not wanted; '64pt' is the height the picture must be resized to, 0.4pt is the thickness of the frame around it (put it to 0pt for no frame) and 'picture' is the name of the picture file
%\quote{Some quote}                                 % optional, remove / comment the line if not wanted

% to show numerical labels in the bibliography (default is to show no labels); only useful if you make citations in your resume
%\makeatletter
%\renewcommand*{\bibliographyitemlabel}{\@biblabel{\arabic{enumiv}}}
%\makeatother
%\renewcommand*{\bibliographyitemlabel}{[\arabic{enumiv}]}% CONSIDER REPLACING THE ABOVE BY THIS

% bibliography with mutiple entries
%\usepackage{multibib}
%\newcites{book,misc}{{Books},{Others}}
  
\newcommand*{\customcventry}[7][.25em]{
  \begin{tabular}{@{}l} 
    {\bfseries #4}
  \end{tabular}
  \hfill% move it to the right
  \begin{tabular}{l@{}}
     {\bfseries #5}
  \end{tabular} \\
  \begin{tabular}{@{}l} 
    {\itshape #3}
  \end{tabular}
  \hfill% move it to the right
  \begin{tabular}{l@{}}
     {\itshape #2}
  \end{tabular}
  \ifx&#7&%
  \else{\\%
    \begin{minipage}{\maincolumnwidth}%
      \small#7%
    \end{minipage}}\fi%
  \par\addvspace{#1}}

\newcommand*{\customcvproject}[4][.25em]{
%   \vfill\noindent
  \begin{tabular}{@{}l} 
    {\bfseries #2}
  \end{tabular}
  \hfill% move it to the right
  \begin{tabular}{l@{}}
     {\itshape #3}
  \end{tabular}
  \ifx&#4&%
  \else{\\%
    \begin{minipage}{\maincolumnwidth}%
      \small#4%
    \end{minipage}}\fi%
  \par\addvspace{#1}}

\setlength{\tabcolsep}{12pt}

%----------------------------------------------------------------------------------
%            content
%----------------------------------------------------------------------------------
\begin{document}
%\begin{CJK*}{UTF8}{gbsn}                          % to typeset your resume in Chinese using CJK
%-----       resume       ---------------------------------------------------------
\makecvtitle
\vspace*{-23mm}

\begin{center}
\begin{tabular}{ c c c }
 \faGlobe\enspace jcken95.github.io & \faEnvelopeO\enspace j.c.kennedy1@ncl.ac.uk & \faGithub\enspace jcken95 \\  
\end{tabular}
\end{center}



\section{EDUCATION}
{\customcventry{September 2022}{Uncertainty Quantification in Complex Energy Systems Models}{PhD Statistics}{Newacstle University, UK}{}{}}
\vspace{0.2cm}
{\customcventry{June 2018}{A Statistical Approach to Product Costing (work with Nissan)}{Newcastle University}{Newacstle University, UK}{}{}}

\section{KEY SKILLS}

{\customcvproject{Statistics}{}
  {\begin{itemize}
    \item {Bayesian Inference \& Applied Statistics \vspace{0.1cm}}
    \item {Gaussian Process Regression \vspace{0.1cm}}
    \item {Uncertainty Quantification \vspace{0.1cm}}
  \end{itemize}
  }
}
\vspace{0.2cm}
{\customcvproject{Programming}{}
  {\begin{itemize}
    \item {R and Rstudio \vspace{0.1cm}}
    \item {MATLAB \vspace{0.1cm}}
    \item {Stan \vspace{0.1cm}}
  \end{itemize}
  }
}
\vspace{0.2cm}
{\customcvproject{Computing}{}
  {\begin{itemize}
    \item {\LaTeX  \vspace{0.1cm}}
    \item {Linux  \vspace{0.1cm}}
    \item {Git \& GitHub  \vspace{0.1cm}}
  \end{itemize}
  }
}

\section{ACADEMIC PROJECTS}

{\customcvproject{PhD Thesis}{Sept 2018 - Ongoing}
  {\begin{itemize}
    \item {Developing novel statistical methodolgy to analyse stochastic computer experiments. \vspace{0.1cm}}
    \item {Working towards performing an a (subjectivist) uncertainty analysis for large offshore windfarms. \vspace{0.1cm}}
  \end{itemize}
  }
}
\vspace{0.2cm}
{\customcvproject{Masters Thesis}{Sept 2017 – May 2018}
{\begin{itemize}
  \item {Collaborated with accountants from Nissan Motor Corporation to enhance the product costing process of their vehicles. \vspace{0.1cm}}
  \item {Applied Machine Learning methods to help accountant detect errors in their large databases. \vspace{0.1cm}}
  \item {Final result included a predictive model to cost vehicles, saving hundreds of hours per year and allowing data to be updated much more frequently. \vspace{0.1cm}}
\end{itemize}
}


}



\section{AWARDS AND ACHIEVEMENTS}
\begin{minipage}{\maincolumnwidth}%
	\small{
    	\begin{itemize}
	  \item {Prize Winner - Best Talks of The Research Students' Conference \vspace{0.1cm}}
          \item {Royal Statistical Society Prize for Outstanding Academic Achievement 2018  \vspace{0.1cm}}
          \item {Chair of Maths \& Stats Student-Staff Committee $2017$--$18$ \vspace{0.1cm}}
		\end{itemize}}%
\end{minipage}%
      

% Publications from a BibTeX file without multibib
%  for numerical labels: \renewcommand{\bibliographyitemlabel}{\@biblabel{\arabic{enumiv}}}% CONSIDER MERGING WITH PREAMBLE PART
%  to redefine the heading string ("Publications"): \renewcommand{\refname}{Articles}
\nocite{*}
\bibliographystyle{plain}
\bibliography{publications}                        % 'publications' is the name of a BibTeX file

% Publications from a BibTeX file using the multibib package
%\section{Publications}
%\nocitebook{book1,book2}
%\bibliographystylebook{plain}
%\bibliographybook{publications}                   % 'publications' is the name of a BibTeX file
%\nocitemisc{misc1,misc2,misc3}
%\bibliographystylemisc{plain}
%\bibliographymisc{publications}                   % 'publications' is the name of a BibTeX file

%-----       letter       ---------------------------------------------------------

\end{document}


%% end of file `template.tex'.
